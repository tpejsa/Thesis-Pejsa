A second evaluation study was conducted to show that attention-signalling properties of gaze shifts can be significantly modified by manipulating the alignment parameters of the gaze shift model. In this study, a virtual character gazed at a series of paintings. We used the torso alignment parameter to control how much the character's upper body participated in these gaze shifts (Figure~\ref{fig:Study3Conditions}). Our expectation was that when the character turned its upper body more toward a painting, this would induce a greater shift in attention in the participant than merely turning the character's eyes and head, so the participants would attribute to the character a greater interest in that painting.

\subsection{Hypotheses}

Our hypotheses for the study were the following:

\begin{enumerate}
\item The character will be perceived as more interested in an object when it gazes at the object than when it ignores the object.
\item The character will be perceived as more interested in an object when it shifts its upper body toward the object, rather than rotating just its eyes and head.
\item The character will be perceived as more interested in an object when it shifts its upper body toward the object with high torso alignment rather than with minimal torso alignment.
\end{enumerate}

The first hypothesis is strongly supported in literature, as the role of eye gaze in attention cueing and establishing joint attention has been extensively studied~\citep{frischen2007gaze}. The latter two hypotheses have support in prior work in the social sciences, which suggests a link between body orientation and human attention cueing mechanisms \citep{hietanen2002social,pomianowska2011socialcues,kendon1990conducting,schegloff1998bodytorque}. On that basis, we hypothesize that increased amount of upper body reorientation in gaze shifts suggests a greater shift in attention.

\subsection{Participants}

We recruited 15 participants (8 males, 7 females) from a campus population, using fliers, online job postings, and in-person recruitment. The study took 10 minutes and participants were paid \$2 for their participation.

\subsection{Study Design}

\begin{figure*}
\centering
\includegraphics[width=0.75\textwidth,page=13]{gazemodel/Figures/tiis14-pejsa.pdf}
\caption{Experimental conditions in Study 3.}
\label{fig:Study3Conditions}
\end{figure*}

The study followed a within-participants design with a single factor with four levels. The experimental conditions were as follows:

\begin{enumerate}
\item \emph{No gaze} -- The character does not look toward the painting.
\item \emph{Eye-head} -- The character gazes at the painting by turning its eyes and head toward the painting, without engaging the torso.
\item \emph{Minimally aligned upper body} -- The character also turns its torso toward the painting by a minimal amount (model parameter $\alpha_T$ set to 0).
\item \emph{Highly aligned upper body} -- The character turns its torso toward the painting by a large amount (model parameter $\alpha_T$ set to 0.3).
\end{enumerate}

Figure~\ref{fig:Study3Conditions} depicts each of the four conditions.

The participant was asked to watch the character perform sequences of gaze shifts toward four virtual paintings placed in the scene. The four paintings were static and fully visible to the participant at all times. For each set of paintings, the character performed three gaze shifts---eye-head gaze, gaze with minimal torso alignment, and gaze with high torso alignment. In each case, the character gazed at a different painting. In any set of paintings, the character would gaze at three out of the four paintings \emph{once} and would never gaze at the fourth painting. The placement of paintings and assignment of gaze conditions were randomized. After watching the gaze shifts, the participant was asked to rate the character's interest in each painting. Figure~\ref{fig:Study3Task}, left shows the task interface, while the figure on the right shows the physical setup.

\begin{figure*}
\centering
\includegraphics[width=1\textwidth,page=14]{gazemodel/Figures/tiis14-pejsa.pdf}
\caption{Experimental task in Study 3. Left: Task interface. Right: An experimenter demonstrating the task.}
\label{fig:Study3Task}
\end{figure*}

The task was built using Unity game engine~\citep{unity3d}. The gaze shift model and task logic were implemented in C\# scripts.

\subsection{Procedure}

After recruitment, each participant read and signed a consent form. Then they performed eight trials of the experimental task. The first three trials were used to acclimate the participant to the task, and data from these trials were not used. At the end of each trial, the participant rated the character's interest in each painting using rating scales that appeared under each painting on the screen (Figure~\ref{fig:Study3Task}, left). After submitting their ratings, a new set of paintings and gaze shifts were presented to the participant. Task duration was approximately five minutes. At the end of the task, each participant filled out a demographic questionnaire.

\subsection{Measures}

The study used one subjective measure---\emph{perceived interest}---which was measured using a single seven-point rating scale item (1 = ``Uninterested,'' 7 = ``Very Interested'').

\subsection{Results}

\begin{figure*}
\centering
\includegraphics[width=0.65\textwidth,page=15]{gazemodel/Figures/tiis14-pejsa.pdf}
\caption{Results for perceived interest ratings for each type of gaze shift. Results for gaze shifts with upper body movement are shown in red.}
\label{fig:Study3Results}
\end{figure*}

Figure~\ref{fig:Study3Results} shows the study results. Mean interest ratings were 1.25 ($\mathrm{SD} = 0.19$) in the \emph{no gaze} condition, 3.67 ($\mathrm{SD} = 0.19$) in the \emph{eye-head} condition, 4.20 ($\mathrm{SD} = 0.19$) in the \emph{minimally aligned upper body} condition, and 5.00 ($\mathrm{SD} = 0.19$) in the \emph{highly aligned upper body} condition. One-way, within-participants ANOVA found significant differences between means, $F(3, 282) = 195.03, p < .0001$.

A priori comparisons showed that the character was perceived as significantly more interested in the \emph{eye-head} condition than in the \emph{no gaze} condition, $F(1, 282) = 218.12, p < .0001$, providing support for the first hypothesis. Furthermore, the character was perceived as significantly more interested in the \emph{minimally aligned upper body} condition than in the \emph{eye-head} condition, $F(1, 282) = 10.65, p = .0012$, leading us to accept the second hypothesis. Finally, the character was perceived as significantly more interested in the \emph{highly aligned upper body} condition than in the \emph{minimally aligned upper body} condition, $F(1, 282) = 23.97, p < .0001$, which supports the third hypothesis.

\subsection{Discussion}

The conducted study shows that changes in upper body orientation in gaze shifts produced by our gaze shift model signal a stronger shift in attention toward objects and information in the environment. Torso rotation amplitude in the \emph{minimally aligned upper body} condition never exceeded 6$^{\circ}$, yet even such a small amount of movement yielded a significant increase in the perceived interest score, suggesting that humans are highly sensitive to differences in body orientation. High torso alignment yielded changes in torso orientation in excess of 26$^{\circ}$ and led to a further increase in interest scores, suggesting a correlation between the character's upper body orientation, controlled using the torso alignment parameter, and the strength of the associated attention cue. 