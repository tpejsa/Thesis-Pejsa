Gaze is a fundamental component of communicatively effective character animation. In animated film and other non-interactive media, a character's gaze can capture the viewer's attention and convey the character's personality and intent, while interactive virtual characters can use gaze cues to enhance their social interactions with users. In all these contexts, the character's nonverbal attending behavior can be realized as a sequence of \emph{directed gaze shifts}---coordinated movements of the eyes, head, torso, and the whole body toward targets in the scene.
The current work introduces new methods for synthesis of directed gaze shifts for animated characters. The methods can synthesize a wide range of gaze movements---from subtle saccades and glances, to whole-body turns. Experiments are presented showing how communicatively effective attending behaviors on virtual agents can be built from these movements.
Furthermore, an approach is introduced for authoring of gaze animation for non-interactive scenarios, which uses directed gaze shifts as authoring primitives. It is empirically demonstrated that the new approach reduces the cost of animation authoring and improves its scalability with respect to scenario complexity and changes in scene layout.
