Gaze is a fundamental component of believable character animation, allowing animated characters to capture the viewer’s attention and direct it toward important objects and information in the environment. The character signals its attention not just with the eyes, but also with coordinated shifts in head and body posture. Producing such richly articulated movements is a challenging problem that requires considerable animator effort and expertise. In this work, I introduce a set of techniques that simplify and automate the process of synthesizing perceptually plausible, communicatively effective gaze. I propose modeling the character’s gaze behavior as \emph{directed gaze shifts}---coordinated, rotational movements of the eyes, head, and body toward targets in the scene. I introduce a kinematic, neurophysiology-based model for synthesis of directed gaze shifts that allows parametric control over head and body posture. I demonstrate how this model enables an animated character to effectively signal the distribution of its attention through shifts in gaze direction and body orientation. I extend the model with a set of techniques for adapting gaze shift kinematics to non-human and stylized character designs. Finally, I present a comprehensive approach for authoring rich gaze behaviors of characters in non-interactive scenarios (such as film), which uses directed gaze shifts as building blocks; I demonstrate how this approach yields plausible, effective gaze animations while greatly reducing the authoring effort required.
