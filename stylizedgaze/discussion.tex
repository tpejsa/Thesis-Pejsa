This chapter describes methods for adaptation of gaze shift motion that allow stylized characters to be animated using models for humanlike gaze shift synthesis. The methods work by adjusting the effective gaze target position and gaze shift kinematics to reduce distracting animation artifacts that would otherwise occur on stylized characters. The methods are demonstrated as extensions of the gaze shift model from Chapter~\ref{cha:GazeShiftModel}, though they can be used in conjunction with any gaze shift synthesis method that allows control over gaze target position and gaze shift timing.

Motion adaptations are applied in a view-dependent way and only allowed to deviate from biological constraints of human gaze when such deviations are undetectable to the viewer. The idea of adapting gaze motion in relation to the camera position in the scene is novel and we use it as a foundation of another method---performative gaze---which allows us to synthesize partially-aligned gaze shifts toward targets in the scene, while maintain the character's alignment---and involvement---with the audience.

A study with human participants confirms that, while our methods depart from accurately representing human biological motion, the resulting gaze behavior is no less accurate than the original, humanlike gaze in communicating attention direction. It is further demonstrated---through examples and objective measurements---that the proposed methods are effective across a range of character designs. Therefore, they contribute to the central objective of this dissertation, which is to facilitate the creation of believable and communicatively effective gaze animation in a scalable way.

