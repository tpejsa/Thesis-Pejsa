The current chapter has presented the design of effective gaze and body reorientation behaviors enabling a virtual agent to reconfigure the conversational formation and signal footing to human users in order to shape their speaking behavior---eliciting conversational contributions from them as addressees or excluding them as bystanders. The presented study with human participants has demonstrated that a virtual agent equipped with these behaviors has the capability to influence users into conforming with their conversational role, which manifests itself specifically in the immersive VR setting. The effectiveness of the proposed behaviors in immersive VR is likely due to participants' increased awareness of the spatial arrangement of the interaction and the agent's gaze cues.

This work motivates the importance of coordinated gaze and body orientation in signaling attention and supporting high-level processes in multiparty interaction with a virtual agent. It also underscores that such capabilities are only going to become more important as consumer VR takes hold and multiparty interactions become an integral part of social VR experiences. However, it only represents a first step in terms of improving agents' nonverbal behaviors to the point of enabling them to autonomously and effectively manage interactions with multiple users within a virtual space. For one, body reorientation is insufficient to effect reconfigurations of the conversational formation as users join and leave the interaction---the agent also needs the ability to move to a different position and account for obstacles in the environment. This would also afford control over the effects of proxemics: as the results of our experiment suggest, inappropriate personal distance can have a negative impact on user experience. Finally, the agent needs to react to user behaviors and infer their engagement intent from their movements---i.e., before engaging with the user as an addressee, it should first understand if that user even desires such a level of engagement or if they prefer to remain on the sidelines as a bystander. Such problems have been explored in the context of physically situated interaction with embodied agents~\citep{bohus2009models}, but they are beyond the scope of the current work.